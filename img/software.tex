% -
 %   Copyright (C) 2015  Beniamine, David <David@Beniamine.net>
 %   Author: Beniamine, David <David@Beniamine.net>
 %   
 %   This program is free software: you can redistribute it and/or modify
 %   it under the terms of the GNU General Public License as published by
 %   the Free Software Foundation, either version 3 of the License, or
 %   (at your option) any later version.
 %   
 %   This program is distributed in the hope that it will be useful,
 %   but WITHOUT ANY WARRANTY; without even the implied warranty of
 %   MERCHANTABILITY or FITNESS FOR A PARTICULAR PURPOSE.  See the
 %   GNU General Public License for more details.
 %   
 %   You should have received a copy of the GNU General Public License
 %   along with this program.  If not, see <http://www.gnu.org/licenses/>.
 %%

\begin{tikzpicture}

    \pic(code) at (.5,0) [draw, fill = gray!30,scale=.4,
    pic text ={
    int main()\{\\\quad printf(``Bom dia'');\\\quad return 0;\\\}}] {computer};
    \node[] at ($(code-c.north)+(0,.5)$) {Code source (human readable)};

    \uncover<2->{

        \pic (bin) [draw, fill = gray!70,scale=.4,
        pic text ={0100010001000101010
                   1110101000010101010
                   0101101000101010101
                   1010100110010111010}]
        at (0:7) {computer};
        \node[] at ($(bin-c.north)+(0,.5)$) {Binary (computer language)};

        \draw[->,thick] (code-c) to[bend left=30] node[above]{compilation} (bin-c);
    }

    \uncover<3->{
        \draw[->,thick] (code-c) to[bend left=30] node[above]{compilation} (bin-c);
    }

    \uncover<4->{
        \draw[->,very thin,dashed] (bin-c) to[bend left=30]  node[above]{reverse engineering} (code-c);
    }

    \uncover<5->{
        \node[fill=red, rounded corners,text centered] (q) at
        ($(code-c.south)+(3.25,1.7)$) {Which version should we distribute ?};
    }
\end{tikzpicture}
